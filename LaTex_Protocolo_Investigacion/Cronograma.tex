\pagebreak
\section{Cronograma}
Las líneas de trabajo están ordenadas por prioridad en la tabla \ref{tab:Cronograma}, por su parte el trabajo 
\textit{Análisis de mediciones de dispersión} y \textit{Algoritmos de Estimación de Distribución} tienen asignado como fecha inicio 
y fin un signo de interrogación ya que no son líneas obligatorias.
%
Sólo se llevarían a cabo, si se consigue desarrollar alguna de las otras líneas en menos tiempo del asignado inicialmente.
%
Nótese que se asignan los primeros dos meses de cada año para redacción de la tesis (además de un periodo final de escritura)
con el fin de ir avanzando en la escritura durante el propio desarrollo de los trabajos.

\begin{table}[H]
\centering
\caption{Cronograma líneas de trabajo} \label{tab:Cronograma}
\resizebox{\textwidth}{!}{%
\begin{tabular}{|c|l|c|}
\hline
\textbf{Fechas inicio} & \textbf{Fecha fin} & \textbf{Tema a trabajar} \\ \hline
\textbf{Enero 2018} & Julio 2019 & Algoritmos basados en dominancia \\ \hline
Agosto 2019 & Diciembre 2019 & Algoritmos basados en descomposición \\ \hline
Diciembre 2019 & Marzo 2019 & Problema de planeación de menú - multi-objetivo \\ \hline
Enero 2020 & Diciembre 2020  & Algoritmos basados en indicadores \\ \hline
Marzo 2020 & Agosto 2020 & Nuevo paradigma \\ \hline
Agosto 2021 & Noviembre 2021 & Optimización de partículas \\ \hline
Septiembre 2019 & Diciembre 2019  & CIANNA \\ \hline
Septiembre 2020 & Enero 2021  & Algoritmo multi-objetivo memético \\ \hline
 Febrero 2021 & Julio 2021 & Escalabilidad en las variables (LSGO) \\ \hline
& ? & Análisis de mediciones de dispersión \\ \hline
& ? & Algoritmos de Estimación de Distribución \\ \hline
Enero 2020 & Febrero 2020  & Redacción \\ \hline
Enero 2021 & Febrero 2021  & Redacción \\ \hline
Octuber 2021 & Febrero 2022  & Redacción \\ \hline
\end{tabular}%
}
\end{table}
