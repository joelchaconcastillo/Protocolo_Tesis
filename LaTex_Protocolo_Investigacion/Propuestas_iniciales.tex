%\section{Propuestas iniciales}

Hasta este momento se han desarrollado varias propuestas iniciales para validar la hipótesis en varias áreas.
%
En particular, se han concretado propuestas en mono-objetivo con una variante de Evolución Diferencial llamada 
''\textit{Differential Evolution with Enhanced Diversity Maintenance}``, en multi-objetivo con 
''\textit{A Dominance-Based Multi-Objective Evolutionary Algorithm with Explicit Variable Space Diversity Managment}``, 
en un problema aplicado con ''\textit{A Novel Memetic Algorithm with Explicit Control of Diversity for the Menu Planning Problem}`` 
y un estudio que relaciona el criterio de parada con los componentes internos del operador de cruce SBX en 
''\textit{Analysis and Enhancement of Simulated Binary Crossover}``.
%
Las primeras tres propuestas se basan en la misma idea básica: incorporar un mecanismo de reemplazo en el cual se 
gestione la diversidad de forma explícita en el espacio de las variables de decisión considerando el criterio de parada y el 
tiempo transcurrido.
%
En cada caso, se ha aplicado a un área diferente (mono-objetivo continuo, mono-objetivo combinatorio y multi-objetivo continuo)
por lo que cada uno de ellos tiene diferencias importantes en cuanto a como adaptar dicho principio al caso específico.
%
La última propuesta incorpora el criterio de parada y el tiempo transcurrido en los mecanismos internos del operador de 
cruce SBX ''\textit{Analysis and Enhancement of Simulated Binary Crossover}``.
%
Para mayor detalle se incluye un documento anexo para cada una de estas alternativas y en este documento, se ofrece
un resumen con las características principales.

\subsection{Descripción de la propuesta basada en evolución diferencial (mono-objetivo)}

En el trabajo de evolución diferencial, principalmente se proponen dos mecanismos novedosos que se incluye
en una evolución diferencial estándar.
%
El primero consiste en una fase de reemplazo en la cual se gestiona la diversidad de los individuos de forma explícita.
%
Esta estrategia relaciona el tiempo transcurrido con el criterio de parada de forma que se evita la convergencia acelerada 
a ciertas regiones del espacio de búsqueda.
%
El segundo mecanismo involucra un archivo de vectores elite con el propósito de mantener a aquellos individuos que 
se encuentran en regiones promisorias, pero que en dicho momento no son suficiente distantes de los individuos actuales como
para ser mantenidos en la población.
%
Dichos soluciones podrían ser reconsideradas en el futuro, cuando los requerimientos de diversidad sean menores.

Los resultados obtenidos fueron notablemente mejor que el estado del arte inclusive considerando otros mecanismos de 
reemplazo para gestionar la diversidad.
%
Una observación en la parte de la validación experimental es que el conjunto de vectores elite mantenían diversidad de 
forma implícita, es decir, que incluso sin forzar la diversidad en los mismos, se mantenían soluciones distantes entre sí.

\subsection{Descripción de la propuesta multi-objetivo}

Aunque se han realizado varios trabajos relacionados con la diversidad en el espacio de las variables del 
ámbito multi-objetivo (ver sección \ref{sec:estado_arte}) ningún trabajo ha alcanzado resultados significativamente superiores 
a otros algoritmos multi-objetivo que no consideran dicha diversidad.
%
Consideramos que uno de los inconvenientes de los métodos en este sentido es que tratan de mantener diversidad todo el tiempo,
lo que puede ser contraproducente.
%
Por ello, se propone el primer método que considera un modelo de preservación de diversidad dinámico lineal donde se toma
en cuenta el tiempo transcurrido y el criterio de parada.
%
A grandes rasgos, en la propuesta se propone una fase de reemplazo similar a la utilizada en evolución diferencial pero al que se
le incluye un estimador de densidad del espacio de los objetivos.
%
Además, la fase de reemplazo en la propuesta multi-objetivo considera la típica ordenación por frentes usados en los algoritmos
basados en dominancia.
%
Por otra parte, el estimador de densidad se basa en la distancia entre el espacio dominado por la soluciones de referencia y 
la solución actual.

\subsection{Descripción de la propuesta aplicada}

En esta propuesta se considera el problema de planeación de menú.
%
Este problema consiste en seleccionar de forma automática un conjunto de platillos (previamente propuestos por expertos) 
para cafeterías escolares.
%
Los platillos están conformados por el plato de entrada, plato fuerte y postre. 
%
Particularmente, este problema es un problema de optimización combinatoria con restricciones.
%
Las restricciones están relacionadas con la cantidad de nutrientes que se deben tomar en las dietas.
%
En este problema se consideran tres aportaciones: se incorpora una fase de reemplazo similar a la aplicada en evolución 
diferencial (\textit{Best Non-Penalized Survivor Strategy - BNP}), se propone un operador de cruce 
(\textit{Similarity Based Crossover - SX}) y se aplica una búsqueda local por escalada (\textit{First-improvement hill climbing}).
%
De estas tres aportaciones, la única esencial para analizar la hipótesis de este trabajo es la primera.
%
Sin embargo, sin incorporar las otras dos modificaciones, sería necesario realizar ejecuciones a muy largo plazo para llegar
a soluciones de alta calidad.
%
En los resultados se observa que basta con realizar ejecuciones a mediano plazo (dos horas) para obtener resultados 
superiores al estado del arte.

\subsection{Descripción propuesta del operador de cruce binario simulado}

En esta parte se describen las modificaciones realizadas para el operador de cruce binario simulado 
''\textit{Simulated Binary Crossover}``, que permitieron generar el operador dinámico de cruce binario simulado 
''\textit{Dynamic Simulated Binary Crossover - DSXB}``.
%
La idea principal de esta propuesta es utilizar el tiempo transcurrido y el criterio de parada para modificar la forma
en que actúa el operador de cruce.
%
Así, se modifican varios aspectos como son la apertura de la curva de distribución o la cantidad de información que se hereda 
de los individuos padre a los individuos hijo sin sufrir modificación.
%
En este estudio no se preserva la diversidad de forma explícita, sin embargo se observa la importancia de considerar el 
criterio de parada y el tiempo transcurrido en los operadores de un algoritmo evolutivo.
