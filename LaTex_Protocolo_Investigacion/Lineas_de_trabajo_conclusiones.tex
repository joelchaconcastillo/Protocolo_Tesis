En este apartado se exponen las distintas líneas que se deben trabajar para el término correcto de la tesis
doctoral.
%
En color azul están aquellas líneas de trabajo que ya están concluidas o casi concluidas.
%
En color gris se presentan un conjunto de líneas no obligatorias. 
%
Estas son líneas de trabajo que podrían aportar conclusiones adicionales, pero que el director de tesis
no considera obligatorias para la validación adecuada de la hipótesis de estudio.
%
En color negro aparecen las líneas de trabajo obligatorias que se deben desarrollar para concluir la tesis
de forma exitosa.
%
La línea CIANNA se refiere a un trabajo de vinculación que se realizará con el 
Centro Integral de Atención a Niñas, Niños y Adolescentes, el cual consiste en un trabajo
similar al de planeación de menús.
%
Dicho proyecto está en espera de ser aprobado por la SICES (Secretaría de Innovación, Ciencia y Educación Superior).
%
En caso de ser aprobado pasaría a ser una línea obligatoria.

    \begin{itemize}
        \item \textbf{Mono-objetivo}
           \begin{itemize}
                \item { \color{blue} Evolución Diferencial - \textbf{99\%}}.
               \item Optimización de Partículas (\textit{PSO})  - \textbf{0\%}.
               \item { \color{gray}Algoritmos de Estimación de Distribución (\textit{EDA})}.
           \end{itemize}
        \item \textbf{Multi-objetivo}
        \begin{itemize}
            \item Algoritmos basados en dominancia.
            \begin{itemize}
                \item Dominancia básico (Carlos Coello) - \textbf{90\%}.
                \item Memético - \textbf{30\%}.
            \end{itemize}
            \item Algoritmos basados en descomposición - \textbf{50\%}.
            \item Algoritmos basados en indicadores - \textbf{20\%}.
            \item Nuevo paradigma (Oliver Schütze) - \textbf{40\%}.
        \end{itemize}
         \item \textbf{Aplicaciones}
         \begin{itemize}
             \item Problema de Planeación de Menú.
              \begin{itemize}
                  \item \textcolor{blue}{Académico (Coromoto - Mono-objetivo) \textbf{100\%}}.
                  \item \textcolor{gray}{CIANNA\footnote{Centro Integral de Atención de Niños, Niñas y Adolescentes} - \textbf{0\%}} 
                  \item Académico (Coromoto - Multi-objetivo) - \textbf{0\%}.
              \end{itemize} 
         \end{itemize}
          \item \textbf{Comunes}
          \begin{itemize}
              \item { \color{blue} Operadores \textbf{100\%}.}
              \item Escalabilidad en las variables ( LSGO ) - \textbf{40\%}.
              \item \textcolor{gray}{Análisis de mediciones de la dispersión.}
          \end{itemize}
    \end{itemize}
