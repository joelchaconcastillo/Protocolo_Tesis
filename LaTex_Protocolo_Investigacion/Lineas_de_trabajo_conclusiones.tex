\pagebreak
\section{Líneas de trabajo}
En este apartado son expuestas las distintas líneas que se deben trabajar para el término de la estancia de doctorado.
%
En color azul están aquellas líneas de trabajo que ya estan concluídas o casi concluídas.
%
En color gris se presentan las líneas no obligatorias, sin embargo es deseable desarrollar.
%
En color negro las líneas obligatorias que se deben trabajar para concluir con la estancia.
%
El proyecto CIANNA (Centro Integral de Atención a Niñas, Niños y Adolescentes) está en espera de ser aprobado por el SICES (Secretaría de Innovación, Ciencia y Educación Superior), el cual consiste en la planeación de menús (similar al MPP).
%
De ser aprobado pasaría a ser una línea obligatoria.

    \begin{itemize}
        \item \textbf{Mono-objetivo}
           \begin{itemize}
                \item { \color{blue} Evolución Diferencial - \textbf{99\%}}.
               \item Optimización de Partículas (\textit{PSO})  - \textbf{0\%}.
               \item { \color{gray}Algoritmos de Estimación de Distribución (\textit{EDA})}.
           \end{itemize}
        \item \textbf{Multi-objetivo}
        \begin{itemize}
            \item Algoritmos basados en dominancia.
            \begin{itemize}
                \item Dominancia básico (Carlos Coello) - \textbf{90\%}.
                \item Memético - \textbf{30\%}.
            \end{itemize}
            \item Algoritmos basados en descomposición - \textbf{50\%}.
            \item Algoritmos basados en indicadores - \textbf{20\%}.
            \item Nuevo paradigma (Oliver Schütze) - \textbf{40\%}.
        \end{itemize}
         \item \textbf{Aplicaciones}
         \begin{itemize}
             \item Problema de Planeación de Menú.
              \begin{itemize}
                  \item \textcolor{blue}{Académico (Coromoto - Mono-objetivo) \textbf{100\%}}.
                  \item \textcolor{gray}{CIANNA\footnote{Centro Integral de Atención de Niños, Niñas y Adolescentes} - \textbf{0\%}} 
                  \item Académico (Coromoto - Multi-objetivo) - \textbf{0\%}.
              \end{itemize} 
         \end{itemize}
          \item \textbf{Comunes}
          \begin{itemize}
              \item { \color{blue} Operadores \textbf{100\%}.}
              \item Escalabilidad en las variables ( LSGO ) - \textbf{40\%}.
              \item \textcolor{gray}{Análisis de mediciones de la dispersión.}
          \end{itemize}
    \end{itemize}