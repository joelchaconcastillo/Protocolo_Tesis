\section{Hipótesis}

Como se ha discutido en los apartados anteriores, en la mayor parte de algoritmos poblacionales es muy difícil de controlar
la velocidad de convergencia y por ello uno de los inconvenientes más comunes a la hora de aplicarlos es que aparezcan problemas
de convergencia prematura, especialmente al realizar ejecuciones a largo plazo.
%
La hipótesis de este trabajo es que se puede superar ampliamente el estado del arte, si a la hora de diseñar las diferentes
componentes de los algoritmos poblacionales se tiene en cuenta tanto el tiempo transcurrido de ejecución como el criterio de parada
para controlar de forma directa o indirecta la diversidad.
%
Además, se hipotetiza que lo anterior es general, en el sentido de que aplica a diversos algoritmos poblacionales (no sólo los 
algoritmos evolutivos) y en diferentes áreas, como puede ser optimización mono-objetivo y multi-objetivo tanto para el caso
continuo como para el caso discreto.
